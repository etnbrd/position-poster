\section*{Abstract}

Real-time web applications need to be highly concurrent, and to scale over a large cluster.
% This is achieved through parallelism.
We identify two types of parallelism to meet such requirements.
% Data parallelism and pipeline parallelism.
Data parallelism replicates the whole logic to process each request in a thread \cite{Behren2003}.
Pipeline parallelism slices logic into stages, each on an independent thread \cite{Ousterhout1996}.
We believe the two designs are on two ends of a design spectrum; and each application requires a specific combination of the two designs.
SEDA \cite{Welsh2000}, Storm and Spark are frameworks to design such combinations.

The two types of parallelism reveals the duality of threads.
% Logical threads and execution threads.
Data parallelism uses a logical thread for each request.
It defines an encapsulation for the logic of processing a request.
Pipeline parallelism uses an execution thread for each core.
It leverages parallelism.
The motives to use logical threads and execution threads are opposite.
We argue that it is a mistake to map the former onto the latter.

Pipeline parallelism implies to manually slice an application for its deployment.
Because it is a manual process, it implies to reduce logical threads to execution threads.
We believe it limits the design and the evolution of a web application.
Behren \textit{et. al.} state that \textit{it is a mistake to attempt high concurrency without help from the compiler} \cite{Behren2003}.
Following this advice, we propose to automate this slicing process.
We expect to improve design flexibility by allowing developers to dissociate logical from execution threads.
% It allows developers to separate logical from execution threads for pipeline parallelism.
For this automation, we resolved two main issues.
Execution distribution and memory distribution.

After decades of improvements, the event and the thread model are now almost alike for a single core \cite{Adya2002}.
Still, the event model conserves a particularity.
Event handlers end without return, they asynchronously trigger the next handler execution.
% Event handlers end without return.
% The next handler asynchronously continues the execution.
% Their executions are independent. % as long as the causality is conserved.
The call stacks of two handlers are disjoints.
Thus, it is possible to distribute the execution on multiple machines, as long as the causality is preserved.

Each distributed handler needs to access the shared memory.
Web applications still heavily rely on single databases to store the whole state.
With recent trends around in-memory caches, we believe it is now viable to bring closer the logic and its state.
We choose to partition and distribute the memory.
We detect the dependencies of each handlers through static analysis.
The handlers communicate through a one way flow of messages to exchange the shared states.

We built a first incomplete compiler as a proof of concept.
After successful results with custom applications, we are now on the process of building a complete compiler to transform real applications.